

\section{System Architecture}

\begin{figure}[ht!]
\centering
\includegraphics[width=1\textwidth]{images/architecture/architecture.png}
\caption{Overall architecture of the system }
\label{overflow}
\end{figure}

The system architecture can be layered into three layers; client, back-end and database. The control flows from the clients requesting a page or inserting data, to the back-end who communicates with the storage. From the storage the execution cycle is reversed. The storage will execute what it’s been asked to do by the back-end and return the result to the back-end. The back-end then again returns the result to the client.

\subsection{Interfaces (1)}

The Soccer Analytic tool has one user interface - the web browser interface. Both the capturing and the analytic interface’s can be reached from this single web browser interface. All interfaces are created on the client. The client will get data from the back-end and construct the interfaces dynamically. 

\subsection{Back-end (2)}

The back-end is the middle layer between the client and the storage. Its main task is to serve static files to the clients, handle data insertions or handling web request by mapping them to database operations. New data insertions are possible inserted into several database indexes. The back-end will then ensure that all indexes are updated before returning success to the client. 

\subsection{Storage (3)}

The storage layers task is to persist data and handle search queries on the data. It consists of several indexes that each stores a part of the domain model. 

\subsection{Domain model}

The domain model is based around matches. All attacks are wrapped into a match root. Attacks can be seen as subdocuments of the match document. Inside the attacks all passes lies with other information. This gives an easy way of understanding how all the data is related together. Below is a complete example of how it all is structured. The number of attacks and passes is stripped down to one in the example.

\lstinputlisting{domainmodel.js}
\label{list:domainmodel}

\begin{figure}[ht!]
\centering
\includegraphics[width=1\textwidth]{images/general/capture_illustration.png}
\caption{Final pitch dividing. More zones have been added to the penalty box area than the initially design. }
\label{fig:finalPitchDividing}
\end{figure}

Figure \ref{fig:finalPitchDividing} shows the final version of the pitch dividing with a illustration of how an attack is captured. Player 4 plays to player 10 from zone 13 to 14. Player 10 transports the ball until he passes to player 15 from zone 18 to 22. Player 15 tries a finish from zone 22 inside the penalty area. The reason for going for more zones in the penalty area was to be able to see more specific where attacks are finished.

\section{Implementation details}

\begin{figure}[ht!]
\centering
\includegraphics[width=1\textwidth]{images/implementation/software_stack.png}
\caption{Software stack}
\end{figure}

\subsection{Storage}

The data storage is an Elasticsearch database\footnote{http://www.elasticsearch.org/}. Elasticsearch is document oriented, schema less and works well with \ac{JSON}\footnotemark. As our server is built on JavaScript, working with \ac{JSON} is easy. \ac{JSON}-objects can be inserted right into the storage and Elasticsearch will map fields and value accordingly, and make it available for search.
Elasticsearch takes advantages of embedded documents meaning we can store related data together. An attack is usually made up of several passes these can be stored as an embedded document in the attack document. Then all passes can be retrieved in one query when fetching an attack.

The main reason for using Elasticsearch is its search capability. In the starting phase of the project MongoDB \footnotemark was the storage engine. However as some aggregation queries were hard to figure out how to do, a change to Elasticsearch was made. It should be noted that MongoDB supports map reduce operations. With some extra work the queries causing problems could possible have been done with MongoDB. However, with Elasticsearch you can in a single, simple to write query get counted how many passes all players for a team has played and received, the number of times all players has been the breakthrough-player in an attack, count type of attacks, count the most used zones for passing and finishing and so on. This makes it very easy and efficient to do queries for analyses on teams and players.

Another feature of Elasticsearch is how easy it is to scale it by adding new nodes to your cluster. It is built to scale horizontally out of the box. Elasticsearch uses a sharding technique to spread data across the cluster.  Basically, it breaks up the data into smaller chunks spreading it across the cluster of machines to achieve scalability and performance. The configuration setup is highly adjustable letting you specify number of shards and replicas the storage shall keep. Adding a new machine to the cluster will automatically start a reorganizing process of the data to take advantage of the extra hardware. 

\footnotetext{http://www.json.org/}
\footnotetext{http://www.mongodb.org/}

\subsubsection{Indexes}

Indexes are much like tables in a relational database in the way that they are a container for data. An index stores documents, which is a bunch of key/value pairs, like \ac{JSON}. You can let Elasticsearch automatic analyze the field data or you can use mappings. For supporting querying on players and other fields where the value is more than one word, you have to tell Elasticsearch that it is a multi field. For example, searching for player name ''Stefan Johansen'' when the field is not set as multi field will give out two results and not one, as you would expect.

In our project we have 5 different indexes. 
\begin{itemize}
\item \emph{Team index}: Holds all the teams.
\item \emph{Player index}: Holds all players for every team.
\item \emph{Match index}: Holds all data from every match, including all attacks for each match.
\item \emph{Attack index}: Holds all attacks in every match.
\item \emph{Pass index}: Holds all passes in every attack.
\end{itemize}

Listing \ref{list:domainmodel} shows an example of data stored from a match. As may be noted there is data redundancy as some indexes stores the same information. One can argue that storage has become so cheap and if you can use a little bit extra space to gain performance you would do it. In this case it was done to be able to fully support aggregation functions on subdocuments.


\subsection{Back-end}

\subsubsection{Overview}	
The back-end is the middleware between the clients and the data layer and is written in Node.js\cite{node.js} It exposes a \ac{REST}\footnote{http://www.ibm.com/developerworks/webservices/library/ws-restful/} interface over \ac{HTTP} shown in figue \ref{fig:api} for the client to communicate. A request coming in is transformed to a database query based on the resource it tries to access. Node.js is built on Chrome’s JavaScript runtime\footnote{https://code.google.com/p/v8/} engine. It’s a relatively new and unproven platform, but has gained a lot of hype in the computer science community for its lightweight and efficient model. This comes from the even-driven and non-blocking I/O model. Per default, an instance of node runs in a single thread and does not utilize other CPU cores. When programming you need to avoid long, blocking computations as this will block the node process for handling new incoming requests. Our back-end uses non-blocking I/O calls when accessing the database and does minimal computation elsewhere. 

\subsubsection{Routing}

The back-end code is structured up in modules. Requests on the URL /players is routed to the Player module for handling. Similar, you have an attack module, pass module, team module, player module and a match module. A very often used URL is /team/:name. A \ac{HTTP} GET on this URL will be routed to the team module where a query for information about the specified team is run. On answer from the database, the result is transformed before returning it to the client. 

Similar, if the client sends new data for a match the server will insert the data into the appropriate indexes. The server will respond will HTTP status code 201 if all goes well or 400 on an error. The server uses \ac{HTTP} code actively to tell the client the result of requests. For all \ac{HTTP} POST requests the server will send status code 201, meaning that a new resource is created. For \ac{HTTP} GET requests the server responds as normal with status code 200, if all went well. 

No URL route uses \ac{HTTP} PUT and DELETE methods. These methods could have been used for deleting or updating attacks or passes. Typically, an operator will do a mistake while capturing new attacks. 

In principle, since the data input is generated in the web browser (with JavaScript), it could have been inserted right away into the database without going through an extra middleware. However, this limits us, as we can’t combine multiple queries by going through the back-end. Cleaning of data before serving to the client would neither be possible. Normally, you also do validation of the data at the back-end before inserting into your database.

\subsubsection{API}

The \ac{API} of the web server is listed in figure \ref{fig:api}. As mentioned the \ac{API} follows the \ac{REST} principles. Optimal, routes like /teams/:name/finalthird should not be needed as all statistics could have been returned when getting team statistics. 

\begin{figure}[ht!]
\centering
\includegraphics[width=1\textwidth]{images/implementation/API.png}
\caption{Overview of the web servers API}
\label{fig:api}
\end{figure}

\subsection{Front-end}

The front end is a \ac{SPA} \cite{} using Backbone.js\footnotemark as under-supporting library. This means that not every code is necessary loaded at once (\ac{HTML}, \ac{CSS} and JavaScript), but when needed. Also the whole page does not reload when users navigate around, only parts of it will be updated. This is all possible because of \ac{AJAX}. \ac{AJAX} is way for JavaScript running in a browser to make HTTP requests to back-end web servers. It’s asynchronously, meaning that the web page is still responsive and navigable during the request\cite{ajax}.

Backbone uses a \ac{MVP} model to structure the code. With Backbone your views will update automatic when data changes. You use Backbones models and presenters (called views in the documentation of Backbone) by letting your own models and presenters inherit Backbones basic classes. Figure \ref{fig:backbone_architecture} shows how the overall architecture of the front-end. In the following sections the architecture of the client and how different concepts is used will be described. 

\footnotetext{http://backbonejs.org/}

\begin{figure}[ht!]
\centering
\includegraphics[width=1\textwidth]{images/architecture/backbone_architecture.png}
\caption{Architecture of the client side. The numerating is used to reference different behavior}
\label{fig:backbone_architecture}
\end{figure}

\subsubsection{Models}

In Backbone you handle data and communication with the back-end with something called models. A model has mainly two responsibilities; whenever an update on a models data occurs the model notifies the presenters that have subscribed for update events for that particular model (4). The second is that models are responsible for \ac{AJAX} communication with the back-end (3). An example is when a user registers a new attack for a match. He fills out a form and press the submit button (1). Then a new attack model is created. Calling save on the instance of the model will send an \ac{AJAX} post request to the server with the models data in the \ac{HTTP} body (3).

Similar to an attack model we have player model that handles everything around players. When you click yourself into a players profile the model will fetch statistics from the back-end, notify the presenter that the data is ready, and the \ac{HTML} will be rendered in the web browser.

There is also a match model for fetching and registration of matches, team model for viewing statistics, and a pass model for saving passes. They work it the same way as the models described above using \ac{AJAX} to communicate with the back-end and notification presenters on data changes.

\subsubsection{Collections}
Collections are a set of models. It has almost the same functionality as a model. You can bind functions to happen on events for any model in the collection. On the client there is two places where collections are used; players and teams. Naturally, players contain many player models and teams collections contain many team models.

\subsubsection{Presenters}

Presenters are the glue between the Backbone models and the view (web page).  Presenters can bind own functions to respond to models events. For example on data change event from a model the presenter will call its render function and the \ac{UI} will updated (4). Each presenter has a own \ac{HTML} div tag which he renders to. Including to subscribing to events on models the presenter can subscribe for view events that happens in the\ac{DOM}, events that the user triggers (1). Following up the example in the previous section on an attack registration. When a user press the submit button the presenter will be notified. He will create a model for the attack and calling the save method on the instance object. 

On initialization of a presenter object it is initialized with a model or a collection depending on the view it should render.  After the model has fetched its data the presenters render function will be called. This is where the data is inserted into the \ref{HTML} template explained in more details in the following sections.

\subsubsection{Views}

For an analytic toolkit to be useful a good presentation is critical. Here several helper libraries are used to present the data. Highcharts.js\footnotemark is a JavaScript library for illustrating graphs. A query on team generates a lot of statistics and rather than listing them up they are presented using charts. Using charts gives us the advantage of displaying several numbers for each player by plotting it in the same graph. In figure \ref{fig:chart} the number of times a player has been involved in all attacks, the number of passes into the final third of the pitch and the number of times a player has been the breakthrough-player is shown.

\footnotetext{http://www.highcharts.com/}

\begin{figure}[ht!]
\centering
\includegraphics[width=100mm]{images/general/chart_passes.png}
\caption{Shows how the passing statistic is illustrated on the client by using Highcharts.js}
\label{fig:chart}
\end{figure}

Positional data is created using the new feature of HTML5, canvas element. It lets you draw graphics on the fly in the web page. In this system it is used to create an element that symbol the different zones in our domain model. The team model will fetch data from the back-end containing numbers for all zones that symbols shots taken from that zone. This is then plotted into the respective zones as figure \ref{fig:attacking_zones} is showing.

\begin{figure}[ht!]
\centering
\includegraphics[width=85mm]{images/general/finishing_zones.png}
\caption{Illustrations of which zones the team has finished their attacks from in percent. Team – \ac{TIL}}
\label{fig:attacking_zones}
\end{figure}

Backbone comes with a library Underscore.js\footnotemark that makes creating HTML pages with dynamic content easily. When you are rendering new content on the site you can insert data retried from a model dynamically into the HTML. In the example below the input is a an array of objects where each element in the array contains a key value pair 'name' : value. The HTML output of this will be a list of names.

\lstinputlisting{underscore_example.html}
\footnotetext{http://underscorejs.org/}

\subsubsection{Router}

A component not mentioned before is the router. The router is the glue that binds all the other Backbone modules together. It handles the navigation between pages in the application. When users navigate on the page by clicking on links or buttons in a traditional web page the back-end serves a new HTML page. Here, a router module handles this click event. The router model examines the URL request and routes the request to the presenter responsible for that specific route. This is mapped inside the router. The presenter will then call fetch functions on the model to get data from the back-end and render the HTML.

In the page the header is static and will only be rendered once when the user first visits the web page. 

\subsection{Building database – players and teams}

Finding squads in a useful format like \ac{XML} or \ac{JSON} was harder than expected. On the Norwegian football alliance's website you could download the squads for the current season, but only in PDF format. Current squads is fetched from altomfotball.no, a website by the Norwegian broadcasting company TV2. The fetching process is an own python script meant to run only once to set up the database. For each team the script basically reads the HTML document with all players listed, parse out players name, and sends in to the web server.

\subsection{Security}
Security is not taken into concern. This means anyone getting into the page can post new match data and add attacks. This could have been fixed by requiring a login before getting access to the site. 







