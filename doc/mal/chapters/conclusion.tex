This chapter presents our achievements, gives some concluding remarks and outlines possible future work.

\section{Achievements}
In this project, we develop and evaluate a system for capturing, persisting and presenting information in the field of soccer opponent analysis. Particularly we have focused on identifying key players in the opponent team. The problem definition is as follow:

\textit{This project will develop a system complementing the Muithu and Bagadus systems. Focus will be on soccer opponent analytics, where a data repository needs to be developed capturing important events relevant for this type of analytics. Especially we want to identify the key players offensive in a team. A user interface component presenting the core information about the opponent should also be developed and evaluated.}

In the requirement specification we stated what we wanted from the system; a complementary opponent analytic system that among other things identifies key players in the offensive play of a soccer team. We wanted to be able to see where on the field the key players contribute. Including to this we wanted an interface for capturing events for soccer opponent analytic and estimate how long the process of capturing these events would take. Lastly, we needed to figure out what to capture from matches.

In the process we have been collaborating with \ac{TIL} to develop a domain model for our system resulting in several rounds of discussions before settling. Some prototyping was done to illustrate different type of statistics the system could provide. 

In order to evaluate the system user serverys was conducted with \ac{TIL} and other experts in the domain of soccer. These assessors has been testing the system with real data from total 12 games where \ac{TIL} and {\SIF} have been involved. Assessors where asked to state how much they agree with different statements on a Likert-scale. From the evaluation results we can conclude that the assessors view the Soccer Analytic Tool as a good system for providing information about soccer opponents. The assessors would have the system rather not having it. Expert assessors ratings variated from neutral to strongly agree on the system as a opponent analytic tool. 

When gathering test data we found out that a typical match takes around 45 minutes to capture with the current interface. We suggested that implementing a complementary interface for selecting players would decrease the time with 20 minutes.

\section{Concluding Remarks}

Building a system for soccer is complex task. There are many different terms in the field making it hard to make a vocabulary for your system. Soccer experts also values information differently and can figure out completely different things from the same statistic. 

\section{Future Work}

Simple quality checks of the data can be added on the back-end to improve accuracy of the data:

\begin{itemize}
\item \emph{Players}: In attacks with more than two passes is register. The attack start player should be the same as the sender of the first pass, or with no passes, the player finishing the attack. 
\item \emph{Player IDs}: Checking that IDs for passes matches a ID for the specified team
\item \emph{Other}: 
\end{itemize}

Including to this, the single match page could have been updated with delete and edit buttons for attacks for operators to be able to correct or delete wrong data.

There is still many possible queries than can be done on the data gathered. Examples include listing what type of passes players play; cross, pass, long ball. The same goes for type of finishes; shot miss, shot target or shot goal. This would need to be discussed with experts in the domain in order to quantify what can be useful. However, there are many possibilities left to explore. 

Giving the end users the possibility to see statistics for two teams at the same time. The statistic for one team itself might not be so interesting if you don't have something to compare it to. Graphical components in the system could have had a way to select a team to compare with.

Another feature would be a team versus team page where two teams are compared up to each other.





