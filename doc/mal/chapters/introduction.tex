\section{Background}

Sports science is an increasingly hot topic and more and more soccer teams are investing time and money into strategic development of big data and sports science \cite{bigdata:majorleague}. All this to increase the player’s performance on the field and reduce the risk of injuries. One of the pioneers in this field for a long time, AC Milan, established in as early as 2002 the MilanLab \cite{bigdata:milanlab}. The goal for the program was to get better and more concrete information about the player’s physical condition by collecting data over time of player’s performance. As AC Milan was one of the first to use big data, the use of big data today has exploded. Services like Opta and Prozone serves player statistics, heat maps, and video analytics of player’s performance. Fans get exposed to these statistics by clubs and broadcasting companies that uses it actively in their coverage of soccer games. The awareness of statistics for clubs and fans has possible never been higher than now \cite{dailymailOnStatistics}.

Soccer teams and athletes in general are constantly looking for possibilities to gain advantages over opponents. In soccer, your next opponent is analyzed down to the smallest details to find weaknesses and strengths. All this to be able to take advantage of your opponents weak points and limit their strengths. It is all about helping the players prepare for matches and let them know what to expect from the opponent team. Knowing about typical team plays and player movement can help you prepare for the match. 

There are several ways to gather information about your opponent; from looking through a full 90 minutes match, to advanced tools, which can highlight key information for you. Some of them are expensive as Prozone \cite{Prozone:indepth}, which is a complete system for capturing and presenting information. For small soccer clubs with relatively small budgets this can be an expensive investment. Another system is Interplay Sports\footnote{http://www.interplay-sports.com/}, which \ac{TIL} uses for game analytic. A video feed from any match can be used as the input, letting you manually tag and describe situations in the match. This can also be done live. In a sport where the next soccer match usually is in 3 days tools for filtering out the useless data is valuable.

In the analytic process there are two processes we will look at; first you need to gather the information and secondly is how to present the information gathered. 

\section{Problem definition}

This project will develop a system complementing the Muithu and Bagadus systems. Focus will be on soccer opponent analytics, where a data repository needs to be developed capturing important events relevant for this type of analytics. Especially we want to identify the key players offensive in a team. A user interface component presenting the core information about the opponent should also be developed and evaluated.

\section{Interpretation}

The project is providing an infrastructure for capturing events like attacks that leads to an attempt on the opponent goal, and presenting this information through a user interface. We are interested in how long it typically takes to capture all relevant events from a single match, as this has to be done manually. A data model that reflects what we want to capture from each attack, needs to be developed for being able to get relevant information for being used as an opponent analytic system.

First, a requirement specification shall be developed. Then a prototype developed that fulfills the requirements. At last specialists in the domain of soccer shall evaluate the system.

The design and development processes will be performed in collaboration with staff of the Norwegian soccer club \ac{TIL}. An end-user test of the system against currently used tools and the system as a opponent analytic system will perform a final evaluation, and the result of this evaluation will be used to conclude the project.

\section{Methodology}

The final report of the ACM Task Force on the Core of Computer Science \cite{computing_as_a_discipline} divides the discipline of computing into three major paradigms:

\begin{itemize}
\item \emph{Theory}: 
Theory: Rooted in mathematics, the approach is to define problems, propose theorems and explore to prove them in order to find new relationships and progress in computing.
\item \emph{Abstraction}: 
Rooted in the experimental scientific method, the approach is to analyze a phenomenon by creating hypothesis, constructing models and simulations, and analyzing the results.
\item \emph{Design}: 
Rooted in engineering, the approach is to state requirements and specifications; design and implement systems that solve the problem, and test the systems in order to find the best solution to the given problem.

\end{itemize}

For this project the design process seems to be the most suitable out of the three paradigms. The design process consists of 4 steps, which are repeated if tests reveal that the latest version of the system does not meet the requirements.

\begin{itemize}
\item \emph{State requirements and specification}: A need or problem is identified, researched, and defined.
\item \emph{Design and implement the system}: Data models and a system architecture are designed. Prototypes are implemented.
\item \emph{Test the system}: Assessment and testing of the aforementioned prototypes.
\end{itemize}

\section{Outline}

\begin{itemize}
\item \emph{Chapter 2}: Presents some related work to the project
\item \emph{Chapter 3}: Describes the requirement specification
\item \emph{Chapter 4}: Describes the design and implementation of the system
\item \emph{Chapter 5}: Gives a demonstration of the system
\item \emph{Chapter 6}: Evaluates and discuss the system
\item \emph{Chapter 7}: Concludes
\end{itemize}





