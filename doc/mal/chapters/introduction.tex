\section{Background}

Soccer teams and athletes are constantly looking for possibilities to gain advantages over opponents. In soccer your next opponent is analyzed down to the smallest details to find weaknesses and strengths. All this to be able to take advantage of your opponents weak points and handle their strengths. An example of a typical weakness is a team playing 4-4-2 and gives up space between the lines (back-four and midfield). An strength can be finding this space, typical with a number 10 player.

There are several ways to gather information about your opponent. From looking through whole matches to advanced tools, which can highlight key information for you.There are two main aspect of the analysis process; First you need to gather the information and secondly is how to present the information, usually for the coaching staff.

\section{Problem definition}

This thesis will develop a system complementing the Muithu and Bagadus systems. Focus will be on soccer opponent analytics, where a data repository need to be developed capturing important events relevant for this type of analytics. Specially we want to identify the breakthrough players. A user interface component providing the core information about the opponent also has to be developed. 

\section{Interpretation}

Our thesis is that providing an infrastructure for capturing events like attacks leading to an attempt on the opponent goal, and presenting this information through an user interface with graphs and illustrations. We are interested in how long it typically takes to manually capture all relevant events from a single match. For this we need to develop a data model that reflects what we want to capture from each attack. This for being able to do relevant queries on the data afterwards and gain insight.

Focus will however mainly be on developing a domain model 

The design and development processes will be performed in collaboration with staff of the Norwegian soccer club Trømsø IL. An end-user comparison of the system against currently available tools will perform a final evaluation, and the result of this evaluation will be used to conclude the thesis.

\section{Methodology}

The final report of the ACM Task Force on the Core of Computer Science \cite{computing_as_a_discipline} divides the discipline of computing into three major paradigms:

\begin{itemize}
\item \emph{Theory}: 
Theory: Rooted in mathematics, the approach is to define problems, propose theorems and seek to prove them in order to determine new relationships and progress in computing.
  
\item \emph{Abstraction}: 
Rooted in the experimental scientific method, the approach is to investigate a phenomenon by stating hypothesis, constructing models and simulations, and analyzing the results.
  
\item \emph{Design}: 
Rooted in engineering, the approach is to state requirements and specifications, design and implement systems that solve the problem, and test the systems to systematically find the best solution to the given problem.

\end{itemize}

For this thesis the design process seems to be the most suitable out of the three paradigms. The design process consist of 4 steps and is expected to be iterated when tests reveal that the latest version of the system does not satisfactorily meet the requirements.

\begin{itemize}
  \item \emph{State requirements and specification}: A need or problem is identified, researched, and defined.
  \item \emph{Design and implement the system}: Data models and a system architecture are designed. Prototypes are implemented.
  \item \emph{Test the system}: Assessment and testing of the aforementioned prototypes.
\end{itemize}

\section{Outline}

\begin{itemize}
  \item \emph{Item 1}: Presents some background information related to the thesis
  \item \emph{Item 2}: Describes the requirement specification
  \item \emph{Item 3}: Describes the general system model
\end{itemize}